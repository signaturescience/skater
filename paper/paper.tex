\documentclass[9pt,a4paper,]{extarticle}

\usepackage{f1000_styles}

\usepackage[pdfborder={0 0 0}]{hyperref}

\usepackage[numbers]{natbib}
\bibliographystyle{unsrtnat}


%% maxwidth is the original width if it is less than linewidth
%% otherwise use linewidth (to make sure the graphics do not exceed the margin)
\makeatletter
\def\maxwidth{ %
  \ifdim\Gin@nat@width>\linewidth
    \linewidth
  \else
    \Gin@nat@width
  \fi
}
\makeatother

\usepackage{color}
\usepackage{fancyvrb}
\newcommand{\VerbBar}{|}
\newcommand{\VERB}{\Verb[commandchars=\\\{\}]}
\DefineVerbatimEnvironment{Highlighting}{Verbatim}{commandchars=\\\{\}}
% Add ',fontsize=\small' for more characters per line
\usepackage{framed}
\definecolor{shadecolor}{RGB}{248,248,248}
\newenvironment{Shaded}{\begin{snugshade}}{\end{snugshade}}
\newcommand{\AlertTok}[1]{\textcolor[rgb]{0.94,0.16,0.16}{#1}}
\newcommand{\AnnotationTok}[1]{\textcolor[rgb]{0.56,0.35,0.01}{\textbf{\textit{#1}}}}
\newcommand{\AttributeTok}[1]{\textcolor[rgb]{0.77,0.63,0.00}{#1}}
\newcommand{\BaseNTok}[1]{\textcolor[rgb]{0.00,0.00,0.81}{#1}}
\newcommand{\BuiltInTok}[1]{#1}
\newcommand{\CharTok}[1]{\textcolor[rgb]{0.31,0.60,0.02}{#1}}
\newcommand{\CommentTok}[1]{\textcolor[rgb]{0.56,0.35,0.01}{\textit{#1}}}
\newcommand{\CommentVarTok}[1]{\textcolor[rgb]{0.56,0.35,0.01}{\textbf{\textit{#1}}}}
\newcommand{\ConstantTok}[1]{\textcolor[rgb]{0.00,0.00,0.00}{#1}}
\newcommand{\ControlFlowTok}[1]{\textcolor[rgb]{0.13,0.29,0.53}{\textbf{#1}}}
\newcommand{\DataTypeTok}[1]{\textcolor[rgb]{0.13,0.29,0.53}{#1}}
\newcommand{\DecValTok}[1]{\textcolor[rgb]{0.00,0.00,0.81}{#1}}
\newcommand{\DocumentationTok}[1]{\textcolor[rgb]{0.56,0.35,0.01}{\textbf{\textit{#1}}}}
\newcommand{\ErrorTok}[1]{\textcolor[rgb]{0.64,0.00,0.00}{\textbf{#1}}}
\newcommand{\ExtensionTok}[1]{#1}
\newcommand{\FloatTok}[1]{\textcolor[rgb]{0.00,0.00,0.81}{#1}}
\newcommand{\FunctionTok}[1]{\textcolor[rgb]{0.00,0.00,0.00}{#1}}
\newcommand{\ImportTok}[1]{#1}
\newcommand{\InformationTok}[1]{\textcolor[rgb]{0.56,0.35,0.01}{\textbf{\textit{#1}}}}
\newcommand{\KeywordTok}[1]{\textcolor[rgb]{0.13,0.29,0.53}{\textbf{#1}}}
\newcommand{\NormalTok}[1]{#1}
\newcommand{\OperatorTok}[1]{\textcolor[rgb]{0.81,0.36,0.00}{\textbf{#1}}}
\newcommand{\OtherTok}[1]{\textcolor[rgb]{0.56,0.35,0.01}{#1}}
\newcommand{\PreprocessorTok}[1]{\textcolor[rgb]{0.56,0.35,0.01}{\textit{#1}}}
\newcommand{\RegionMarkerTok}[1]{#1}
\newcommand{\SpecialCharTok}[1]{\textcolor[rgb]{0.00,0.00,0.00}{#1}}
\newcommand{\SpecialStringTok}[1]{\textcolor[rgb]{0.31,0.60,0.02}{#1}}
\newcommand{\StringTok}[1]{\textcolor[rgb]{0.31,0.60,0.02}{#1}}
\newcommand{\VariableTok}[1]{\textcolor[rgb]{0.00,0.00,0.00}{#1}}
\newcommand{\VerbatimStringTok}[1]{\textcolor[rgb]{0.31,0.60,0.02}{#1}}
\newcommand{\WarningTok}[1]{\textcolor[rgb]{0.56,0.35,0.01}{\textbf{\textit{#1}}}}

% disable code chunks background
%\renewenvironment{Shaded}{}{}

% disable section numbers
\setcounter{secnumdepth}{0}

%% added by MLS, this is not in the F1000 style by default %%

\hypersetup{unicode=true,
            pdftitle={skater: An R package for SNP-based Kinship Analysis, Testing, and Evaluation},
            pdfkeywords={bioinformatics, kinship, R, genealogy, SNPs, single nucleotide polymorphisms, relatedness},
            pdfborder={0 0 0},
            breaklinks=true}

%% End added by MLS %%

\setlength{\parindent}{0pt}
\setlength{\parskip}{6pt plus 2pt minus 1pt}



\begin{document}
\pagestyle{front}

\title{\textbf{skater}: An R package for SNP-based Kinship Analysis, Testing, and Evaluation}

\author[1]{Stephen D. Turner}
\author[1]{V. P. Nagraj}
\author[1]{Matthew Scholz}
\author[1]{Shakeel Jessa}
\author[1]{Carlos Acevedo}
\author[2]{Jianye Ge}
\author[2]{August E. Woerner}
\author[2]{Bruce Budowle}
\affil[1]{Signature Science, LLC., Austin, TX 78759, USA.}
\affil[2]{Center for Human Identification, Department of Microbiology, Immunology, and Genetics, University of North Texas Health Science Center, Fort Worth, TX 76107, USA.}

\maketitle
\thispagestyle{front}

\begin{abstract}
\hfill\break
\textbf{Motivation:} SNP-based kinship analysis with genome-wide relationship estimation and IBD segment analysis methods produces results that often require further downstream processing and manipulation. A dedicated software package that consistently and intuitively implements this analysis functionality is needed.\\
\textbf{Results:} Here we present the skater R package for \textbf{S}NP-based \textbf{k}inship \textbf{a}nalysis, \textbf{t}esting, and \textbf{e}valuation with \textbf{R}. The skater package contains a suite of well-documented tools for importing, parsing, and analyzing pedigree data, performing relationship degree inference, benchmarking relationship degree classification, and summarizing IBD segment data.\\
\textbf{Availability:} The skater package is implemented as an R package and is released under the MIT license at https://github.com/signaturescience/skater. Documentation is available at https://signaturescience.github.io/skater.
\end{abstract}

\section*{Keywords}
bioinformatics, kinship, R, genealogy, SNPs, single nucleotide polymorphisms, relatedness


\clearpage
\pagestyle{main}

\textbf{R version}: R version 4.0.4 (2021-02-15)

\textbf{skater package version}: 0.1.0

\begin{center}\rule{0.5\linewidth}{0.5pt}\end{center}

\hypertarget{introduction}{%
\section{Introduction}\label{introduction}}

Inferring familial relationships between individuals using genetic data is a common practice in population genetics, medical genetics, and forensics. There are multiple approaches to estimating relatedness between samples, including genome-wide measures, such as those implemented in Plink \citep{purcell2007} or KING \citep{manichaikul2010}, and methods that rely on identity by descent (IBD) segment detection, such as GERMLINE \citep{gusev2009}, hap-IBD \citep{zhou2020}, and IBIS \citep{seidman2020}. Recent efforts focusing on benchmarking these methods \citep{ramstetter2017} have been aided by tools for simulating pedigrees and genome-wide SNP data \citep{caballero2019}. Analyzing results from genome-wide SNP-based kinship analysis or comparing analyses to simulated data for benchmarking have to this point required writing one-off analysis functions or utility scripts that are seldom distributed with robust documentation, test suites, or narrative examples of usage. There is a need in the field for a well-documented software package with a consistent design and API that contains functions to assist with downstream manipulation, benchmarking, and analysis of SNP-based kinship assessment methods. Here we present the skater package for \textbf{S}NP-based \textbf{k}inship \textbf{a}nalysis, \textbf{t}esting, and \textbf{e}valuation with \textbf{R}.

\hypertarget{methods}{%
\section{Methods}\label{methods}}

\hypertarget{implementation}{%
\subsection{Implementation}\label{implementation}}

The skater package provides an intuitive collection of analysis and utility functions for SNP-based kinship analysis. Functions in the package include tools for importing, parsing, and analyzing pedigree data, performing relationship degree inference, benchmarking relationship degree classification, and summarizing IBD segment data, described in full in the \emph{Use Cases} section below. The package adheres to ``tidy'' data analysis principles, and builds upon the tools released under the tidyverse R ecosystem \citep{Wickham2019}.

The skater package is hosted in the Comprehensive R Archive Network (CRAN) which is the main repository for R packages: \url{http://CRAN.R-project.org/package=skater}. Users can install skater in R by executing the following code:

\begin{Shaded}
\begin{Highlighting}[]
\FunctionTok{install.packages}\NormalTok{(}\StringTok{"skater"}\NormalTok{)}
\end{Highlighting}
\end{Shaded}

Alternatively, the development version of skater is available on GitHub at \url{https://github.com/signaturescience/skater}. The development version may contain new features which are not yet available in the version hosted on CRAN. This version can be installed using the \texttt{install\_github()} function in the devtools package:

\begin{Shaded}
\begin{Highlighting}[]
\FunctionTok{install.packages}\NormalTok{(}\StringTok{"devtools"}\NormalTok{)}
\NormalTok{devtools}\SpecialCharTok{::}\FunctionTok{install\_github}\NormalTok{(}\StringTok{"signaturescience/skater"}\NormalTok{, }\AttributeTok{build\_vignettes=}\ConstantTok{TRUE}\NormalTok{)}
\end{Highlighting}
\end{Shaded}

When installing skater, other packages which skater depends on are automatically installed, including magritr, tibble, dplyr, tidyr, readr, purrr, kinship2, corrr, rlang, and others.

\hypertarget{operation}{%
\subsection{Operation}\label{operation}}

Minimal system requirements for installing and using skater include R (version 3.0.0 or higher) and several tidyverse packages \citep{Wickham2019} that many R users will already have installed. Use cases are demonstrated in detail below. In summary, the skater package has functions for:

\begin{itemize}
\item
  Reading in various output files produced by commonly used tools in SNP-based kinship analysis
\item
  Pedigree parsing, manpulation, and analysis
\item
  Relationship degree inference
\item
  Benchmarking and assessing relationship classification accuracy
\item
  IBD segment analysis post-processing
\end{itemize}

A comprehensive reference for all the functions in the skater package is available at \url{https://signaturescience.github.io/skater/}.

\hypertarget{use-cases}{%
\section{Use Cases}\label{use-cases}}

\hypertarget{pedigree-parsing-manipulation-and-analysis}{%
\subsection{Pedigree parsing, manipulation, and analysis}\label{pedigree-parsing-manipulation-and-analysis}}

The skater package has several functions for importing, parsing, and analyzing pedigree data. Pedigrees define familial relationships in a hierarchical structure. Many genomics tools for working with pedigrees start with a .fam file, which is a tabular format with one row per individual and columns for unique IDs of the mother, father, and the family unit. The skater package contains the function \texttt{read\_fam()} to read in a PLINK-formatted .fam file and another function \texttt{fam2ped()} to convert the content into a pedigree object as a nested tibble with one row per family. All pedigree processing from skater internally leverages a data structure from the kinship2 package \citep{sinnwell2014}. Further functions such as \texttt{plot\_pedigree()} produce a multi-page PDF drawing a diagram of the pedigree for each family, while \texttt{ped2kinpair()} produces a pairwise list of relationships between all individuals in the data with the expected kinship coefficients for each pair (see skater package vignette).

\hypertarget{relationship-degree-inference-and-benchmarking}{%
\subsection{Relationship degree inference and benchmarking}\label{relationship-degree-inference-and-benchmarking}}

The skater package includes functions to translate kinship coefficients to relationship degrees. The kinship coefficients could come from \texttt{ped2kinpair()} or other kinship estimation software.

The \texttt{dibble()} function creates a \textbf{d}egree \textbf{i}nference t\textbf{ibble}, with degrees up to the specified maximum degree resolution, expected kinship coefficient, and lower and upper inference ranges as defined in \citet{manichaikul2010}. The \texttt{kin2degree()} function infers the relationship degree given a kinship coefficient and a maximum degree resolution (e.g., 7th-degree relatives) up to which anything more distant is classified as unrelated.

Once estimated kinship is converted to degree, it may be of interest to compare the inferred degree to known degrees of relatedness. When aggregated over many relationships and inferences, this can help benchmark performance of a particular kinship analysis method. The skater package adapts a \texttt{confusion\_matrix()} function from \citet{clark2021} to provide standard contingency table metrics (e.g.~sensitivity, specificity, PPV, precision, recall, F1, etc.) with a new reciprocal RMSE (R-RMSE) metric. The R-RMSE metric is defined more thoroughly in the skater package vignette and may be a preferable measure of classification accuracy when benchmarking relationship degree estimation. In many kinship benchmarking analyses, classification error is treated in a categorical manner (exact match plus or minus one degree), neglecting the true amount of sharing as a real number. Taking the reciprocal of the target and predicted degree in a typical RMSE calculation results in larger penalties for more egregious misclassifications (e.g., classifying a first-degree relative pair as second-degree) than misclassifications at more distant relationships (e.g., classifying a fourth-degree relative pair as fifth-degree).

\hypertarget{ibd-segment-analysis}{%
\subsection{IBD segment analysis}\label{ibd-segment-analysis}}

Tools such as hap-IBD \citep{zhou2020}, and IBIS \citep{seidman2020} detect shared IBD segments between individuals. The skater package includes functionality to take those IBD segments, compute shared genomic centimorgan (cM) length, and converts that shared cM to a kinship coefficient. In addition to inferred segments, these functions can estimate ``truth'' kinship from simulated IBD segments \citep{caballero2019}. The \texttt{read\_ibd()} function reads pairwise IBD segments from IBD inference tools and from simulated IBD segments. The \texttt{read\_map()} function reads in genetic map in a standard format which is required to translate the total centimorgans shared IBD to a kinship coefficient using the \texttt{ibd2kin()} function.

\hypertarget{summary}{%
\section{Summary}\label{summary}}

The skater R package provides a robust software package for data import, manipulation, and analysis tasks typically encountered when working with SNP-based kinship analysis tools. All package functions are internally documented with examples, and the package contains a vignette demonstrating usage, inputs, outputs, and interpretation of all key functions. The package contains internal tests that are automatically run with continuous integration via GitHub Actions whenever the package code is updated. The skater package is permissively licensed (MIT) and is easily extensible to accommodate outputs from new genome-wide relatedness and IBD segment methods as they become available.

\hypertarget{software-availability}{%
\section{Software availability}\label{software-availability}}

\begin{enumerate}
\def\labelenumi{\arabic{enumi}.}
\item
  Software available from: \url{http://CRAN.R-project.org/package=skater}.
\item
  Source code available from: \url{https://github.com/signaturescience/skater}.
\item
  Archived source code at time of publication: \textbf{FIXME Zenodo DOI to come here}.
\item
  Software license: MIT License.
\end{enumerate}

\hypertarget{author-information}{%
\section{Author information}\label{author-information}}

SDT and VPN developed the R package.

All authors contributed to method development.

SDT wrote the first draft of the manuscript.

All authors assisted with manuscript revision.

All authors read and approved the final manuscript.

\hypertarget{competing-interests}{%
\section{Competing interests}\label{competing-interests}}

No competing interests were disclosed.

\hypertarget{grant-information}{%
\section{Grant information}\label{grant-information}}

This work was supported in part by award 2019-DU-BX-0046 (Dense DNA Data for Enhanced Missing Persons Identification) to B.B., awarded by the National Institute of Justice, Office of Justice Programs, U.S. Department of Justice and by internal funds from the Center for Human Identification. The opinions, findings, and conclusions or recommendations expressed are those of the authors and do not necessarily reflect those of the U.S. Department of Justice.

{\small\bibliography{bibliography.bib}}

\end{document}
